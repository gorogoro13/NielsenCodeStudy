%\documentclass[11pt,a4paper]{jsarticle}
\documentclass[11pt,a4paper]{jreport}
%
% ######## measure #########
% # mm = 1mm = 2.85pt      #
% # cm = 10mm = 28.5pt     #
% # in = 25.4mm = 72.27pt  #
% # pt = 0.35mm = 1pt      #
% # em = width of [M]      #
% # ex = height of [x]     #
% # zw = width of [Kanji]  #
% # zh = height of [Kanji] #
% ##########################
% ##################### Portrait Setting #########################
% # TOP = 1inch + ¥voffset + ¥topmargin + ¥headheight + ¥headsep #
% #     = 1inch + 0pt + 4pt + 20pt + 18pt (default)              #
% # BOTTOM = ¥paperheight - TOP -¥textheight                     #
% ################################################################
\setlength{\textheight}{\paperheight}   % 紙面縦幅を本文領域にする(BOTTOM=-TOP)
\setlength{\topmargin}{4.6truemm}       % 上の余白を30mm(=1inch+4.6mm)に
\addtolength{\topmargin}{-\headheight}  % 
\addtolength{\topmargin}{-\headsep}     % ヘッダの分だけ本文領域を移動させる
\addtolength{\textheight}{-60truemm}    % 下の余白も30mm(BOTTOM=-TOPだから+TOP+30mm)
% #################### Landscape Setting #######################
% # LEFT = 1inch + ¥hoffset + ¥oddsidemargin (¥evensidemargin) #
% #      = 1inch + 0pt + 0pt                                   #
% # RIGHT = ¥paperwidth - LEFT - ¥textwidth                    #
% ##############################################################
\setlength{\textwidth}{\paperwidth}     % 紙面横幅を本文領域にする(RIGHT=-LEFT)
\setlength{\oddsidemargin}{-0.4truemm}  % 左の余白を25mm(=1inch-0.4mm)に
\setlength{\evensidemargin}{-0.4truemm} % 
\addtolength{\textwidth}{-50truemm}     % 右の余白も25mm(RIGHT=-LEFT)
%
\title{ニューラルネットワークと深層学習}
\author{著者: Michael Nielsen}

\date{\today}


\begin{document}
\maketitle
%
%
\chapter{第1章の見出し}
\begin{verbatim}
 "network.py"のトレースと解説をせよ。
実行は
net = network.Network([784, 30, 10])
net.SGD(training_data, 30, 10, 3.0)
としたものとする。
\end{verbatim}
%%%%%%%%%%%%%%%%%%%%%%%%%%%%%%%%%%%%%%%%%%%%%
\section{net = network.Network([784, 30, 10])}
   :  (第1章第1節の本文)
   ここは第1段落.ここは第1段落.ここは第1段落.
ここは第1段落.ここは第1段落.ここは第1段落.ここは第1段落.
ここは第1段落.ここは第1段落.

ここは第2段落.ここは第2段落.ここは第2段落.ここは第2段落.
ここは第2段落.ここに2cmの空白→\hspace{2cm}←ここに2cmの空白.
ここは第2段落.ここは第2段落.ここは第2段落.
\subsection{第1章第1節第1項の見出し}
   :  (第1章第1節第1項の本文)
\section{第1章第2節の見出し}
   :  (第1章第2節の本文)
   \begin{itemize}
    \item 一つ目の項目
    \item 二つ目の項目
    \item 三つ目の項目
   \end{itemize}
\begin{enumerate}
\item 一つ目の項目
\item 二つ目の項目
\item 三つ目の項目
\item 四つ目の項目
\end{enumerate}
\begin{description}
\item[札幌ラーメン] 麺は太めのちぢれ麺であり,
    醤油,塩,味噌などのバリエーションがあるが,
    特に味噌ラーメンが重視されている.
    具は一般的なものに加えてコーンを加えることが多い.
    バターをのせることもある.
\item[博多ラーメン] 細いストレート麺が主流で,
    中でも長浜ラーメンの麺はかなり細い.
    スープはトンコツが主体で,
    色の白いスープが多い.
    かなり脂っこいものもある.
    付け合わせに紅しょうが・からし高菜を用いるようだ.
\item[和歌山ラーメン] 細目の麺だが,
    博多ラーメンほどではない.
    わりと伸び気味にゆでるせいか,
    スープで麺が太っていることもある.
    スープはトンコツベースの醤油味.
    いっしょに早すし(早熟れ鮨,あせ巻き寿司)を食べる.
\end{description}
\begin{quotation}
ここは引用文の1つ目の段落.ここは引用文の1つ目の段落.
ここは引用文の1つ目の段落.ここは引用文の1つ目の段落.

ここは引用文の2つ目の段落.ここは引用文の2つ目の段落.
ここは引用文の2つ目の段落.ここは引用文の2つ目の段落.
\end{quotation}
\begin{quote}
ここは引用文の1行目の文章.ここは引用文の1行目の文章.
ここは引用文の1行目の文章.ここは引用文の1行目の文章.

ここは引用文の2行目の文章.ここは引用文の2行目の文章.
ここは引用文の2行目の文章.ここは引用文の2行目の文章.
\end{quote}
\begin{center}
この部分は \\
中央に \\
揃えられます.
\end{center}
この部分は詰め込みが行われる本文.
この部分は詰め込みが行われる本文.
\begin{verbatim}
    ○┃×┃×
    ━╋━╋━
    ×┃○┃○
    ━╋━╋━
    ○┃○┃×
\end{verbatim}
この部分は詰め込みが行われる本文.
\verb+そのまま表示する文字列+
この部分は詰め込みが行われる本文.
\[
\left(
\begin{array}{lcr}
a     & x     & 1    \\
a+b   & x+y   & 10   \\
a+b+c & x+y+z & 100
\end{array}
\right)
\]
\begin{tabular}{p{3cm}cr}
ラーメン & 並 & 500 円 \\
ラーメン & 大 & 600 円 \\
チャーシューメン & 並 & 600 円 \\
チャーシューメン & 大 & 700 円
\end{tabular}
\begin{tabular}{||p{3cm}|c|r||} \hline
ラーメン & 並 & 500 円 \\ \hline
ラーメン & 大 & 600 円 \\ \hline
チャーシューメン & 並 & 600 円 \\ \hline
チャーシューメン & 大 & 700 円  \\ \hline
\end{tabular}
\begin{tabular}{|p{3cm}|c||r|} \hline
\multicolumn{2}{|c||}{メニュー} & \multicolumn{1}{c|}{価格} \\ \hline
ラーメン & 並 & 500 円 \\ \hline
ラーメン & 大 & 600 円 \\ \hline
チャーシューメン & 並 & 600 円 \\ \hline
チャーシューメン & 大 & 700 円  \\ \hline
\end{tabular}
\end{document} 
